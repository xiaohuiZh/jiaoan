% !TEX root = ../jiaoAnthesis.tex
\chapter{等比数列的概念}
\TM 人教 A 版(2019),选择性必修第二册\par
\TC 第四章,数列, 4.3.1 等比数列的概念\par
\LT 新授课\par
\TP \par
\begin{enumerateout}
    \item 通过生活中的实例,理解等比数列的概念和通项公式的意义,了解等比中项的概念.
    \item 体会等比数列与指数函数的关系.
    \item 通过等比数列的概念、通项公式认识等比数列的性质.
\end{enumerateout}
\TE 等比数列、等比中项的概念、等比数列的通项公式、等比数列的性质、等比数列的应用.\par
\TD 等比数列的运算、等比数列的性质及应用.\par
% \TMethods 问题探索法及启发式讲授法\par
\TProcess
\section{概念引入}
\begin{problem}
前面我们学习了等差数列,类比等差数列的研究思路和方法,从运算的角度出发,你觉得还有怎样的数列是值得研究的?
\end{problem}

\TSA 学生独立思考、讨论交流.

教师提示,类比已有的学习经验是一个好方法,比如“等差数列”;然后指引学生回顾等差数列相邻两项的关系,确定新数列的研究问题:相邻两项比是固定常数.

\DP 意在引导学生从运算的角度,类比已有研究对象的主要特征,发现一个新的特殊数列作为研究对象,这样的过程有利于培养学生发现问题和提出问题的能力.
\begin{problem}
“请看下面几个问题中的数列”,类比等差数列的研究,你认为可以通过怎样的运算发现以上数列的取值规律?你发现了什么规律?
\begin{enumerateout}
    \item $9,9^{2}, 9^{3}, 9^{4}, 9^{5}, 9^{6}, 9^{7}, 9^{8}$
    \item $\frac{1}{2}, \frac{1}{4}, \frac{1}{8}, \frac{1}{16} \ldots$
    \item $2,4,8,16,32,64, \ldots$
\end{enumerateout}
\end{problem}
\TSA 学生独立观察,充分思考,交流讨论.

根据学生交流讨论情况,教师可以适时地选择以下问题进行追问.

\reproblem
\begin{enumerateout}
    \item 你能用自然语言归纳每组数列的特征吗?(从相邻两项间的关系分析)
    \item 请归纳概括上述三个具体例子的共同特点. (类比等差数列的过程)
    \item 类比等差数列的概念,从上述几个数列的规律中,你能抽象出等比数列的概念吗?可以用符号语言表示吗?
\end{enumerateout}

\TSA 教师引导学生梳理观察、讨论、分析的结果,抽象概括成数学定义,给出等比数列的定义.

\DP 让学生充分经历从观察、分析到抽象、概括的过程,其中包括独立思考和交流讨论.这是一个提升学生数学抽象素养的时机.

\section{概念理解}
\begin{problem}
结合等比数列的定义,观察等比数列的相邻三项,你有什么新的发现?
\end{problem}
\TSA 让学生独立阅读这段内容,然后分别提出自己的新发现.

教师根据学生的回答情况,可以选择以下问题进行追问 .

\reproblem
\begin{enumerateout}
    \item 等比数列相邻三项有什么代数关系?
    \item 类比等差中项,你能得到等比中项的定义吗?能够用符号语言表示吗?
\end{enumerateout}


\TSA 根据学生探究的情况,教师引导,帮助学生建立等比中项的定义.

\DP 对于难度不大的内容,引导学生通过类比的方法去找到等比数列中相邻三项的关系,并抽象概念得到等比数列的定义.
\begin{problem}\label{pblm:a}
你能等比数列的定义推导它的通项公式吗?
\end{problem}

\TSA 让学生先独立思考,教师展示学生推导并规范解答.

\DP 内容难度不大,引导学生类比等差数列通项公式的推导过程进行推导,并得到等比数学的通项公式.这是一个提升学生数学抽象的时机.
\begin{problem}
    在等差数列中,公差$d\neq 0$的等差数列可以与相应的一次函数建立联系,通过类比,等比数列可以与那个函数建立联系?单调性如何?这里让学生“类比指数函数的性质,说明公比$q$的等比数列的单调性”.
\end{problem}

\TSA 学生独立思考、讨论交流.

教师提示,类比指数函数的性质,说明公比$q>0$的等比数列的单调性.

\DP 让学生充分经历从观察、分析的过程,其中包括独立思考和交流讨论.

\begin{problem}
    在等差数列中,如果$p+q=s+t$,则有$a_p+a_q=a_s+a_t$ ,等比数列有类似的性质么?
\end{problem} 

\TSA 引导学生观察思考,类比等差数列的性质,推导等比数列的性质.

\DP 通过等差数列与等比数列之间的联系,把等差数列的一些性质迁移到等比数列中,发展学生的数学抽象,数学运算、逻辑推理的数学素养.

\section{概念巩固与应用}
\begin{example} \label{exa:1}
    若等比数列$\{a_n\}$的第4项和第6项分别为 48和 12,求$\{a_n\}$ 的第5 项.
\end{example}
\TSA 学生分析解题思路,给出解答并讨论交流,教师进行展示总结.

\DP  例\ref{exa:1}与4.2节的例7类似,也给出了两个独立的条件.
根据两个给定条件得到的关于首项$a_1$和公比$q$ 的方程组的解法往往不唯一,
有时会得到两个$q$ 的值,也就是得到两个不同的等比数列.此例题可以让学生掌握分类讨论的方法.
例\ref{exa:1}也可以直接利用等比中项的定义进行解决,鼓励学生从多角度思考问题.

\begin{example}
    已知等比数列 $\{a_n\}$的公比为 $q$,试用$\{a_n\}$ 的第$m$ 项$a_m$ 表示$a_n$ .
\end{example}
\TSA 学生独立思考,教师给出解答示范.

\DP 等比数列通项公式的应用,给你两个条件$a_1$ 与$q$ 可以表示数列的每一项,同时等比数列的任意一项都可以由数列的某一项和公比表示.
\begin{example}
 数列 $\{a_n\}$共有5项,前三项成等比数列,后三项成等差数列,第3项等于80,第2项与第4项的和等于136,第1项与第5项的和等于132.求这个数列.
\end{example}
\TSA 学生独立思考,教师给出解答示范.

\DP 例3安排了一道综合应用等差数列和等比数列的通项公式解决问题的题目.根据条件包含的等量关系,列出关于数列相关量的方程组是解决这类问题的常用策略.本题利用中间量去表示其他各项,可以减少所设未知数的个数.通过此题提高学生分析问题、解决问题的能力.

\begin{example}
     已知数列$\{a_n\}$ 的首项 .
\begin{enumerateout}
    \item 若 $\{a_n\}$为等差数列,公差$d=2$ ,证明数列 $\{3^{a_n}\}$为等比数列;
    \item 若$\{a_n\}$ 为等比数列,公比$\frac{1}{9}$ ,证明数列 $\{  \log_{3}{a_n} \}$为等差数列.
\end{enumerateout}
\end{example}

\TSA 学生独立思考,教师给出解答示范.

\DP 通过典型例题,加深对等差与等比数列概念的理解,体会等差与等比数列的内在联系.发展学生逻辑推理,直观想象、数学抽象和数学运算的核心素养.

\section{课堂小结}
本节课学习了等比数列、等比中项的概念、等比数列的通项公式、等比数列的性质.
\section{课后作业}
\HW 完成本节课课后习题.

